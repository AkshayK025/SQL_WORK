*use db_cte and table employee_performance*

🟢 Level 1 — Basics
Problem 1
Create a CTE that calculates:
emp_id
emp_name
department
average performance score per employee
Then select all employees whose average score is greater than 85.

Problem 2
Using a CTE:
Count how many projects each employee has worked on
Display employees who worked on more than 1 project

Multiple CTEs
Problem 3
Using two CTEs:
First CTE → calculate average salary per department
Second CTE → list employees with their salary
Final query → show employees earning above department average

Problem 4
Using a CTE:
Calculate maximum performance score per department
Show only employees whose score equals the department maximum
(Hint: CTE + join back to table)


🟠 Level 3 — Multiple CTEs
Problem 5
Using two CTEs:
First CTE → calculate average salary per department
Second CTE → list employees with their salary
Final query → show employees earning above department average

Problem 6
Create a CTE that:
Assigns a performance grade using CASE:
A → score ≥ 90
B → 80–89
C → below 80
Then:
Display count of employees per grade

Level 4 — Advanced (Interview Favorites)
Problem 7
Using a CTE:
Rank employees by performance_score within each department
Display only the top performer per department
(Hint: window function)

Problem 8
Using a CTE:
Find employees whose performance improved over time
(latest score > earliest score)
(Hint: MIN(date), MAX(date))

Problem 9
Create a recursive CTE that:
Generates numbers from 1 to 10
Then join it with the dataset to:
Repeat each employee row based on that number (for practice)